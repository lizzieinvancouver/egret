\documentclass{article}[12pt]
\usepackage{Sweave}
\usepackage{mdframed}
\usepackage{hyperref}
\topmargin -1.5cm        
\oddsidemargin -0.04cm   
\evensidemargin -0.04cm
\textwidth 16.59cm
\textheight 21.94cm 
\renewcommand{\baselinestretch}{2}
\parindent 0pt
\usepackage{setspace}
\singlespacing

\newmdenv[
  topline=true,
  bottomline=true,
  skipabove=\topsep,
  skipbelow=\topsep
]{siderules}


\begin{document}
\section*{Scraping new papers for EGRET: August 5, 2024}
\subsection*{Getting Started: copied from ospree howtoscrape2019.tex}
You'll need:
\begin{enumerate}
\item Excel or other program that makes .xls  or .csv files
\item ImageJ download for free from here \url{https://imagej.net/Welcome}. You'll also need to add the Figure\textunderscore Calibration.class, which will help for giving x and y calibrations to images. 
\begin{enumerate}
\item To add the the Figure\textunderscore Calibration.class: In ImageJ go to plugins, select \textit{add plug in}, then navigate to the Figure\textunderscore Calibration.class file that you downloaded on your computer, click on it and follow through a few clicks to add the plugin.
\item If you have some trouble getting the measurements to show up after calibrating, try switching to the pointer and clicking. You might need to set the preferences on your pointer tool to auto-measure.
\begin{enumerate}
\item A clarification on above from Tim Savas (original OPSREE lead data enterer): \textit{After doing the figure calibration and selecting the yellow pointer tool, you start clicking inside the figure but no points appear. The reason for this is that the "rectangle" you drew for the figure calibration is still masking the figure, and until you click out of it, you can't draw points under it. It's hard to see! So to get rid of the invisible rectangle, just click the mouse once outside of its edge. Side note: After drawing all of your points onto a figure, you can press Command-M to bring up the resulting table of values. I do this in the video tutorial, and whenever I scrape, but didn't describe the key command!}
\end{enumerate}
\end{enumerate}
\end{enumerate}

Now here's what to do:
\begin{enumerate}
\item Copy the excel file data\egret.xlsx for git repo (egret/data) and make your own extension using your initials, for example, Jane Doe would write ``egret\textunderscore dmb.xlxs". This will be the spread sheet you enter your data into and then in the future, someone will save the tab as a csv and merge all of our files together into the master data.
\item Familiarize yourself with each tab:
  \begin{enumerate}
  \item \textbf{meta\_general}: metadata for each sheet
  \item \textbf{source}: list of the paper we are working with. Bibliographic information and notes on usefulness for our purposes. Note the "assigned.to" column, which tells you which figure or table to focus on. You may find upon reviewing the text the paper does not suite out selection criteria, in which case you would change the `A' (accept) in your accept_reject column to `R' for reject. This tab should also be updated if a paper is not published in english.
  \item \textbf{data\_detailed}: Detailed data for the experiment, with all relevant information filled out.
  \item \textbf{scratch}: For temporary formatting and manipulating data scraped from ImageJ.
  \item The two most important tabs to fill out are \textbf{source} and \textbf{data\_detailed}. 
  \end{enumerate}
\item Read your paper and update out the information in the ``source" column if needed and  fill the ``data\textunderscore detailed tab." Be sure your datasetID and study info agrees with the source tab, if not---figure out what is wrong and fix it.
\item Read the paper to decide if the study is eligible for inclusion in EGRET. These selection criteria include:
\begin{itemize}
\item Seeds are germinated under experimental conditions
\item Not conducted under natural conditions in the field 
\item Species are not crops
\item Experimental treatments are not confounded
\end{itemize}
\item Take a screen shot of the figure and import into ImageJ, following detailed instructions from Tim's OSPREE data scraping video from git/ospree/notes/howtoscrape/Data Scraping Tutorial.mp4 or the general instructions outlined in the egret wiki. Use the scratch tab to get data into the right format, and then copy into data\textunderscore detailed.\end{enumerate}
\subsection*{A few more ``how to's", trouble shooting etc} 
 \begin{itemize}
 \item \textbf{For dealing with lats and lons:} Tool to convert to decimal latitude and longitude:\url{https://andrew.hedges.name/experiments/convert_lat_long/}  and remember to add NEGATIVE to your longitude if it's West. Also, you can check where things are by just typing in lat and long into Google maps.
\iitem \textbf{On entering response times:} If the figure is a time-series or curve of the percent germination over time, the days to germination will be the values from the figures's x-axis. Otherwise the value will be the final day that germination was observed. Remember for time-series to also include the zero point, or to include days along the x-axis that have zero on the y-axis as this is still part of the curve. 
\item \textbf{On dealing with error:} If it's figures records error and it is *clear enough* to scrape, error can be recorded. Often times the SE bars are in the way of each other or not quite discernible, in which case we've decided to avoid them. But if the bars are clear, record them. Values can go in "resp\textundersccore error" and just SE in "error type."
\end{itemize}

\section*{Data cleaning and decisions}
\begin{enumerate}
\item Cleaning code are located in the analysis/cleaning/source folder in the egret repo.
\item The following outlines some of the more complex cleaning decisions and issues:
\item \textbf{cleanCoordinates.R}: A number of studies do not give specific lat and longs. To infer the coordinates for these studies, we used Googel Earth to drop a pin in the centre of the named region (county, city, etc) and get a general value for these studies. 
\item \textbf{cleanGerminationTempDuration.R}: Germination temperatures in some studies varied between day and night, studies with multiple temperatures were manually checked and additional data columns entered for day and night temperatures. 
\item \textbf{cleanChillTempDuration.R}: There are 30 studies that are missing chilling temperature. These pdf's were manually checked and notes made to check whether this is true or improperly entered data. It was discovered that in some of these studies, there are no chilling treatments, but there is moderately cold storage temperatures. We discussed whether this storage could function as chilling, but decided that it depended on whether the storage is wet, which would allow the seeds to be imbibed.
\item \textbf{cleanPhotoperiod.R}: 
\item \textbf{cleanStorage.R}: The cleaning of storage is now linked to chilling and whether studies without chilling have moist storage.
\item \textbf{cleanYearGermination.R}: Over 100 studies do not specify what year the study was conducted. It was discussed and our meeting Aug 2 as to whether it was worth our manually checking the paper. It was decided that it was not, although if we do change our mind, it would be possible for a number of studies to infer it based on the language used in some papers (e.g. seeds were stored for x months after collection").

\end{enumerate}


\end{document}
